
\newif\ifshowsolutions
\showsolutionstrue
\input{preamble}
\newcommand{\boldline}[1]{\underline{\textbf{#1}}}

\usepackage{graphicx}
\graphicspath{ {images/} }

\chead{%
  {\vbox{%
      \vspace{2mm}
      \large
      Machine Learning \& Data Mining \hfill
      Caltech CS/CNS/EE 155 \hfill \\[1pt]
      Miniproject 3\hfill
      Released March $3^{rd}$, 2017 \\
    }
  }
}

\begin{document}
\pagestyle{fancy}

% LaTeX is simple if you have a good template to work with! To use this document, simply fill in your text where we have indicated. To write mathematical notation in a fancy style, just write the notation inside enclosing $dollar signs$.

% For example:
% $y = x^2 + 2x + 1$

% For help with LaTeX, please feel free to see a TA!

%%%%%%%%%%%%%%%%%%%%%%%%%%%%%%%%%%%%%%%%%%%%%%%%%%%%%%%%%%%%%%%%%%%%%%%%%%%%%%%%%%%%%%%%

\section{Introduction}
\medskip
\begin{itemize}
    \item \textbf{Group members:} Enrico Borba, Claire Goeckner-Wald
    \item \textbf{Team name:} Papa Mart's Mini Gary - The End
    \item \textbf{Division of labour:}
        Enrico Borba: Programming, ideas, report visualization.
        Claire Goeckner-Wald: Programming, ideas, report assembly.
\end{itemize}


%%%%%%%%%%%%%%%%%%%%%%%%%%%%%%%%%%%%%%%%%%%%%%%%%%%%%%%%%%%%%%%%%%%%%%%%%%%%%%%%%%%%%%%%
\section{Basic Visualizations}
\medskip
% - Justify your choice of visualization method
% - What did you observe?
% - Did the results match what you would expect to see?
% - How do the ratings of the best movies compare to those of those of the most popular movies
% - How do the ratings of the three genres you chose compare to one another
% - Any other comparisons/observations

\begin{itemize}
  \item \textbf{Choice of visualization method:} We used a line graph to show visualize the dataset. We didn't use histograms, because the amount of data shown on the chart would have overwhelmed the reader using this method. For visualizing our three genres, we used the size of the dot to indicate the number of genres the movie is in (larger dot implies more genre crossover); the color of the dot indicated the standard deviation of the reviews (yellow implies higher standard deviation, blue implies lower standard deviation.)
  \item \textbf{Observations:} We observed that 3 was not the average rating, perhaps unsurprisingly. While one might expect a uniform, or normal distribution, center around 3 stars, 4 stars was in fact the most common rating given. On second thought, this is perhaps more expected, because people will tend to watch movies they enjoy - so that would skew the distribution towards higher ratings.
  \begin{center}
    \includegraphics[scale=.6]{basic_1.png}
  \end{center}
  \item \textbf{Results:} The results approximately matched what we expected.
  \item \textbf{Best ratings:} With the exception of Star Wars, none of the top 10 movies with the highest number of ratings overlapped with the top 10 highest average-rated movies. This is perhaps unexpected. To visualize the highest average-rating movies, we decided to use the size of the line to indicate the number of ratings for the film. This visualization method really shows how much Star Wars stands out. In order to acquire this list, we did not include films with fewer than 50 ratings, which removed films receiving one/two five star rating (and thus
  having a very high average of five stars.)
  \begin{center}
    \includegraphics[scale=.5]{basic_2.png} \\
    \includegraphics[scale=.4]{basic_3.png}
  \end{center}
  \item \textbf{Three genres:} We note that the SciFi genre has no movies rated over 4 stars with less than about 200 ratings. We postulate that this is because SciFi's are more difficult to produce, and thus the better scifi films (those with funding) would correlate with more/better advertising (and thus more ratings). As a reminder, we used the size of the dot to indicate the number of genres the movie is in (larger dot implies more genre crossover); the color of the dot indicated the standard deviation of the reviews (yellow implies higher standard deviation, blue implies lower standard deviation.)
  \begin{center}
    \includegraphics[scale=.3]{basic_4_romance.png} 
    \includegraphics[scale=.3]{basic_4_scifi.png}
    \includegraphics[scale=.3]{basic_4_thriller.png}
  \end{center}
\end{itemize}


%%%%%%%%%%%%%%%%%%%%%%%%%%%%%%%%%%%%%%%%%%%%%%%%%%%%%%%%%%%%%%%%%%%%%%%%%%%%%%%%%%%%%%%%
\section{Matrix Factorization Algorithm}
\medskip
% - What parameters did you adjust and how?
% - Justify your choices for the parameters and stopping criteria
% - Did you make any other significant modifications or additions
\begin{itemize}
  \item \textbf{Adjustment of parameters:}
    We used an $\eta$ of $0.01$, and regularization constant of
    $\lambda = 10^{-3}$.
  \item \textbf{Justification of parameters and stopping criteria:}
    We attempted a grid search with $\eta=0.1, 0.01, 0.001$, and
    $\lambda = 10^{-2}, 10^{-3}, 10^{-4}$. And found that the lowest loss was
    encountered at $\eta=0.01$ and $\lambda = 10^{-3}$. We chose to center the
    grid search at $\eta=0.01$ and $\lambda = 10^{-3}$ because those were the
    parameters we used for $K=20$ on assignment 6.
  \item \textbf{Significant modifications:}
    No significant modifications were made. We used the homework solutions as
    our code to obtain $U$ and $V$.
\end{itemize}


%%%%%%%%%%%%%%%%%%%%%%%%%%%%%%%%%%%%%%%%%%%%%%%%%%%%%%%%%%%%%%%%%%%%%%%%%%%%%%%%%%%%%%%%
\section{Marix Factorization Visualization}
\medskip
% - What did you observe?
% - How do the visualizations of the best movies compare to those of the most popular movies
% - How do the visualizations of the three genres you chose compare to one another
% - What was expected and what was surprising from the visualizations?
% - Any other comparisons/observations
\begin{itemize}
  \item \textbf{Observations:} 
  \item \textbf{Best ratings:}
  \item \textbf{Three genres:}
  \item \textbf{Expected, and unexpected:}
\end{itemize}



%%%%%%%%%%%%%%%%%%%%%%%%%%%%%%%%%%%%%%%%%%%%%%%%%%%%%%%%%%%%%%%%%%%%%%%%%%%%%%%%%%%%%%%%
\section{Conclusions}
\medskip
% - Briefly summarize your main observations
% - Did your visualizations help you to better understand the MovieLens dataset?
\begin{itemize}
  \item \textbf{Summary:}
  \item \textbf{Did it help?:} The visualizations helped us better understand the MovieLens dataset. 
\end{itemize}


\end{document}